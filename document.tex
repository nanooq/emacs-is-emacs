\documentclass[]{beamer}
\usepackage[utf8]{inputenc}
\usepackage[T1]{fontenc}
\usepackage{lmodern}

%\setbeameroption{show notes}
%\setbeameroption{hide notes}
%\setbeameroption{show only notes}

%\setbeamercovered{transparent}
%\setbeamertemplate{navigation symbols}{}

\usepackage[ngerman]{babel}
\usepackage{csquotes}
\usetheme{Darmstadt}

\usepackage{graphicx}
\usepackage{subcaption}
\usepackage{hyperref}

\author{nanooq}
\title{emacs is emacs}
\subtitle{A Workshop for danb}
%\logo{}
\institute{Hackspace Siegen e.~V.\\
https://github.com/nanooq/emacs-is-emacs\\
}
\date{2018-XX-XX XX:XX \\  Somewhere}
\subject{emacs workshop for danb}

\begin{document}
	\frame[plain]{\maketitle}
	
	\begin{frame}
		\frametitle{Inhalt}
		\tableofcontents
	\end{frame}

	\begin{frame}
		\section{Teil 1: Foo}
		\frametitle{Teil 1: Foo} 
		\framesubtitle{foooooo}
			\begin{figure}[h!]
			\includegraphics[width=0.5\textwidth]{images/emacs}
			\caption{Logo}
			\end{figure}
	\end{frame}
 
 	\begin{frame}
 	\subsection{Teil 1.1: Bar}
 	\frametitle{Teil 1.1: Bar}  
 	\framesubtitle{baaaar}
 	Solche Treffen in der Vormorderne werden als Herrschertreffen bezeichnet \dots
 	\note[item]{Das Herrschertreffen oder die Herrscherbegegnung ist ein Fachbegriff der Geschichtswissenschaft und bezeichnet persönliche Zusammentreffen von Monarchen als Mittel der Politik. Für die Treffen von Staats- und Regierungschefs hat sich im 20. Jahrhundert der Begriff Gipfeltreffen etabliert \cite{wiki:herrschertreffen}.}
	 \end{frame}
 

\end{document}
